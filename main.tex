% BEWARE OF SPAGHETTI CODE!!!

\documentclass{article}
\usepackage{graphicx}
\usepackage{geometry}
\usepackage{tabularx}
\usepackage{multirow}
\usepackage{float}
\usepackage{amsmath}
\usepackage{amssymb}
\usepackage{textcomp}
\usepackage{dsfont}

\geometry{
    top = 0.1 cm,
    bottom = 0.5 cm,
    left = 0.3 cm,
    right = 0.1 cm
}

% notazione Densa per le radici a denominatore
% (ovviamente va usato in un math environment)
\newcommand{\ngrt}[2][]{
    \sqrt[\mathbf{-}#1]{#2}
}

% modo facile per scrivere bra e ket di cose
% (\left e \right servono per l'autosize delle parentesi)
\newcommand{\bra}[1]{
    \left\langle #1 \right|
}
\newcommand{\ket}[1]{
    \left| #1 \right\rangle
}
\newcommand{\bkprod}[2]{
    \left\langle #1 | #2 \right\rangle
}

% operatore identità (a quanto pare \mathbb{1} non dà quello che ci si aspetterebbe)
\newcommand{\id}{
    \mathds{1}
}

\begin{document}

\footnotesize
\noindent Notazione: $ \ngrt{x} := \frac{1}{\sqrt{x}} $

\noindent
\begin{tabular}{cccccc}
    \multicolumn{2}{c|}{Basi generalizzate} & \multicolumn{3}{c|}{Esiti misure e probabilità (Principio 4)} \\
    $\ket{x} = \xi_x(x) = \delta(x-x_0) $ & \multicolumn{1}{c|}{$\bkprod{x_0}{x_0'} = \delta(x_0 - x_0') $} & $w(a_k) = \frac{\left|\bkprod{a_k}{\psi} \right|^2}{||\psi||^2} $ & $w(a_k) =  \sum_{i=1}^{d_k}\frac{\left|\bkprod{a_{k,i}}{\psi} \right|^2}{||\psi||^2} $ & \multicolumn{1}{c|}{$\mathrm{d}w(a) = \rho(a)\mathrm{d}a =  \frac{\left| \bkprod{a}{\psi} \right|^2}{||\psi ||^2}$} \\
    $\ket{p} = v_p (x) = \ngrt{2\pi\hbar}\,e^{\frac{ipx}{\hbar}} $ & \multicolumn{1}{c|}{$\bkprod{p_0}{p_0'} = \delta(p_0 - p_0') $} & $ \ket{\psi} = \sum_{k=1}^N c_k\ket{a_k} $ & $ \ket{\psi} = \sum_{k=1}^N\sum_{i=1}^{d_k} c^i_k\ket{a_k} $ & \multicolumn{1}{c|}{$\ket{\psi} =  \int\mathrm{d}a\, c(a)\ket{a} $} \\
    \cline{1-2} \cline{6-6}
    Principio 6 & \multicolumn{1}{c|}{$\ket{\psi}\rightarrow\ket{\psi'} =  \frac{P_{a_k}\ket{\psi}}{\sqrt{\bra{\psi}P_{a_k}\ket{\psi}}} $} & $ w(a_k) = \frac{\left|c_k\right|^2}{||\psi||^2}$ & $ w(a_k) = \sum_{i=1}^{d_k} \frac{\left|c^i_k\right|^2}{||\psi||^2}$ & \multicolumn{1}{c|}{$\rho(a) =  \frac{\left| c(a) \right|^2}{||\psi||^2} $} & $P_{a_k} = \ket{a_k}\bra{a_k} $ \\
    \cline{3-5} \\ \hline
\end{tabular}

\begin{minipage}{0.798\linewidth}
    \noindent Per trovare una base di autovettori comuni (sapendo già che gli operatori commutano, e se entrambi hanno degenerazioni):
    \begin{enumerate}
        \item Trovo autovalori e autovettori di $A$ e $B$;
        \item Autovettori associati ad autovalori \textbf{non} degeneri sono automaticamente autovettori comuni;
        \item Per autovettori associati ad autovalori degeneri, faccio la prova (applico $B$ a un autovettore degenere di $A$);
        \item Se è anche autovettore di $B$, sono a posto (è autovettore comune);
        \item Se non lo è:
        \begin{enumerate}
            \item Definisco un nuovo vettore come combinazione lineare degli autovettori della base dell'autospazio degenere in questione;
            \item Impongo che questo nuovo vettore sia autovettore di $B$;
            \item Risolvo il sistema di equazioni trovando i coefficienti della combinazione lineare;
            \item Per come è stato definito, questo vettore è autovettore sia di $A$ che di $B$.
        \end{enumerate}
    \end{enumerate}

    \noindent Per capire se un insieme di osservabili compatibili costituisce un ICOC:
    \begin{enumerate}
        \item Se gli osservabili sono compatibili, esiste una base comune di autovettori;
        \item A ogni autovettore, associo una \textit{label} costituita da una lista dei corrispondenti autovalori per ogni osservabile;
        \item Se \textbf{ogni} \textit{label} è unica, l'insieme è un ICOC.
    \end{enumerate}
\end{minipage}
\hfill
\begin{minipage}{0.173\linewidth}
    \begin{tabular}{|c}
        Matrici di Pauli \\
        $\sigma_x = \left(\begin{matrix}
            0 & 1 \\
            1 & 0
        \end{matrix}\right) $ \\
        $\sigma_y = \left(\begin{matrix}
            0 & -i \\
            i & 0
        \end{matrix}\right) $ \\
        $\sigma_z = \left(\begin{matrix}
            1 & 0 \\
            0 & -1
        \end{matrix}\right) $ \\ 
        $\sigma_x^2 = \sigma_y^2 = \sigma_z^2 = $ \\
        $= -i\sigma_x\sigma_y\sigma_z = \id $ \\
        $\sigma_i\sigma_j = \delta_ij + i\epsilon_{ijk}\sigma_k $ \\ \hline
        Spin \\
        $\Vec{S} = \frac{\hbar}{2}\Vec{\sigma} $ \\ \hline
        Trasformazione unitaria \\
        $A' = UAU^\dagger $ \\ \hline
    \end{tabular}
\end{minipage}

\noindent
\begin{tabular}{cccccc}
    \hline
    Equazione di Schrödinger & \multicolumn{1}{c|}{Visuale di Schrödinger} & Equazione di Heisenberg & \multicolumn{2}{c|}{Visuale di Heisenberg} & Sistema conservativo \\
    $ -i\hbar\frac{\mathrm{d}}{\mathrm{d}t}\ket{\psi(t)} = H(t)\ket{\psi(t)} $ & \multicolumn{1}{c|}{$\begin{cases}\ket{\psi(t)}_S = U(\Delta t) \ket{\psi(t_0)}_S \\ A_S (t) = A_S (0) \end{cases} $} & $ i\hbar\frac{\mathrm{d}}{\mathrm{d}t}A_H(t) = [A_H, H] $ & \multicolumn{2}{c|}{$\begin{cases}\ket{\psi(t)}_H = \ket{\psi(t_0)}_H \\ A_H(t) = U^\dagger(\Delta t)A_H(t_0)U(\Delta t) \end{cases} $} & $ U(t,t_0) = e^{-\frac{i}{\hbar}H(t-t_0)} $ \\
    \hline
    Matrice densità & Stato puro & Stato misto & \multicolumn{2}{c|}{Proprietà generali} & $N\ket{n} = n\ket{n} $ \\
    $\rho(t) = \ket{\psi(t)}\bra{\psi(t)} $ & $\rho^2(t) = \rho(t) $ & $\rho(t) = \sum_k p_k\rho_k(t) $ & $\rho^\dagger(t) = \rho(t) $ & \multicolumn{1}{c|}{$\langle A \rangle_\psi(t) = Tr(\rho(t) A) $} & $a\ket{n} = \sqrt{n}\ket{n-1} $ \\
    \cline{3-3}
    \multicolumn{2}{l|}{$\rho_{pn}(t) = \bra{u_p}\rho(t)\ket{u_n} = \bar{c}_n(t)c_p(t) $} & \multicolumn{1}{c|}{Oscillatore armonico} & $Tr(\rho(t)) = 1 $ & \multicolumn{1}{c|}{$i\hbar\frac{\mathrm{d}\rho(t)}{\mathrm{d}t} = [H(t), \rho(t)] $} & $a^\dagger\ket{n} = \sqrt{n+1}\ket{n+1} $ \\
    \cline{1-2} \cline{4-5}
    \multicolumn{2}{c|}{Condizioni al contorno buche di potenziale} & $H = \hbar\omega\left(N+\frac{1}{2}\right) $ & $\hat{X} := \sqrt{\frac{m\omega}{\hbar}}X $ & $a = \ngrt{2}(\hat{X}+i\hat{P}) $ & $[a, a^\dagger] = 1 $ \\
    Continuità di $\psi$ nelle & \multicolumn{1}{c|}{Continuità di $\psi' $ nelle} & $N = a^\dagger a $ & $\hat{P} := \ngrt{m\hbar\omega}P $ & $a^\dagger = \ngrt{2}(\hat{X}-i\hat{P}) $ & $[N, a^\dagger] = a^\dagger $ \\
    discontinuità di V & \multicolumn{1}{c|}{discontinuità \textbf{finite} di V} & \multicolumn{3}{r}{$u_n(x) = \left[\frac{1}{n!2^n}\left(\frac{\hbar}{m\omega}\right)^n\right]^\frac{1}{2} \left(\frac{m\omega}{\pi\hbar}\right)^\frac{1}{4}\left[\frac{m\omega}{\hbar}x-\frac{\mathrm{d}}{\mathrm{d}x}\right]^n e^{-\frac{m\omega}{\hbar}\frac{x^2}{2}} $} & $[N, a] = -a $ \\
    \cline{3-6}
    \multicolumn{2}{c|}{Soluzioni buche di potenziale ($A, B \in \mathbb{C}$)} & Metodo perturbativo & \multicolumn{3}{l}{$E_n^{(1)} = \bra{n^{(0)}}\hat{W}\ket{n^{(0)}}\qquad \ket{n^{(1)}} = -\sum_{k\neq n} \frac{\bra{k^{(0)}}\hat{W}\ket{n^{(0)}}}{E_k^{(0)} - E_n^{(0)}}\ket{k^{(0)}} $} \\
    \cline{3-3}
    $E > V: $ & $E = V $ & \multicolumn{1}{c|}{$E < V $} & \multicolumn{3}{c}{\multirow{3}{200pt}{$E_n^{(2)} = \bra{n^{(0)}}\hat{W}\ket{n^{(1)}} = -\sum_{k\neq n} \frac{\left|\bra{k^{(0)}}\hat{W}\ket{n^{(0)}}\right|^2}{E_k^{(0)} - E_n^{(0)}} $}} \\
    $\psi(x) = Ae^{ikx} + Be^{-ikx} $ & $\psi(x) = A + Bx $ & \multicolumn{1}{c|}{$\psi(x) = Ae^{\rho x} + Be^{-\rho x} $} \\
    \cline{2-2}
    \multicolumn{1}{c|}{$k := \sqrt{\frac{2m(E-V)}{\hbar^2}} $} & \multicolumn{1}{c|}{Operatori diff in coord sferiche} & \multicolumn{1}{c|}{$\rho := \sqrt{\frac{2m(V-E)}{\hbar^2}} $} \\
    \cline{1-1} \cline{3-6}
    
\end{tabular}

\newpage \newpage

\section*{Stati}

\begin{tabular}{cccc}
    Principio 1 & Funzione d'onda e densità di probabilità & Trasformata di Fourier & Basi generalizzate \\
    $\mathcal{S} \mapsto \mathcal{H} $ & $P(x) =  \frac{|\psi(x)|^2}{||\psi(x)||^2} $ & $\widetilde{\psi}(p) =  \ngrt{2\pi\hbar}\int\mathrm{d}x\psi(x)e^{-\frac{ipx}{\hbar}} $ & $\ket{x} = \xi_x(x) = \delta(x-x_0) $ \\
    $\Sigma \mapsto \hat{\psi} := \{\lambda\ket{\psi} |\, \lambda\in\mathbb{C}\backslash\{0\} \} $ & $P(x) \geq 0,\quad  \int \mathrm{d}x P(x) = 1 $ & $P(p) =  \frac{|\psi(p)|^2}{||\psi(p)||^2} $ & $\ket{p} = v_p (x) = \ngrt{2\pi\hbar}\,e^{\frac{ipx}{\hbar}} $ \\
     &  &  & $\bkprod{x_0}{x_0'} = \delta(x_0 - x_0') $ \\
     &  &  & $\bkprod{p_0}{p_0'} = \delta(p_0 - p_0') $
\end{tabular}

\section*{Osservabili}

\begin{tabular}{cccccc}
    Posizione e impulso & Principio 2 & Principio 3 & \multicolumn{3}{c}{Principio 4} \\
    $X\psi(x) = x\psi(x) $ & $\mathcal{A}\mapsto A $ & $A\ket{a} = a\ket{a} $ & $w(a_k) = \frac{\left|\bkprod{a_k}{\psi} \right|^2}{||\psi||^2} $ & $w(a_k) =  \sum_{i=1}^{d_k}\frac{\left|\bkprod{a_{k,i}}{\psi} \right|^2}{||\psi||^2} $ & $\mathrm{d}w(a) = \rho(a)\mathrm{d}a =  \frac{\left| \bkprod{a}{\psi} \right|^2}{||\psi ||^2}$ \\
    $P\psi(x) =  -i\hbar\frac{\mathrm{d}\psi(x)}{\mathrm{d}x} $ & $\langle \mathcal{A}\rangle_\Sigma = \frac{\bra{\psi}A\ket{\psi}}{\bkprod{\psi}{\psi}} $ & $\sigma(\mathcal{A}) = \sigma(A) $ & $ \ket{\psi} = \sum_{k=1}^N c_k\ket{a_k} $ & $ \ket{\psi} = \sum_{k=1}^N\sum_{i=1}^{d_k} c^i_k\ket{a_k} $ & $\ket{\psi} =  \int\mathrm{d}a\, c(a)\ket{a} $ \\
    $[X,P] = i\hbar$ & $\Delta A = \sqrt{\langle A^2 \rangle - \langle A \rangle^2} $ &  & $ w(a_k) = \frac{\left|c_k\right|^2}{||\psi||^2}$ & $ w(a_k) = \sum_{i=1}^{d_k} \frac{\left|c^i_k\right|^2}{||\psi||^2}$ & $\rho(a) =  \frac{\left| c(a) \right|^2}{||\psi||^2} $
\end{tabular}

\section*{Proiettori e misure}

\begin{tabular}{ccccc}
    Definizione di proiettore $(P, \mathcal{D}(P))$ & Principio 6 & Osservabili compatibili \\
    $P$ è proiettore $\Leftrightarrow P^\dagger = P \wedge P^2 = P $ & $\ket{\psi}\rightarrow\ket{\psi'} =  \frac{P_{a_k}\ket{\psi}}{\sqrt{\bra{\psi}P_{a_k}\ket{\psi}}} $ & $\mathcal{A,B} $ compatibili $\Leftrightarrow [A,B] = 0 $ \\
    $P_q =  \sum_i \ket{\varphi_i}\bra{\varphi_i} $, $\{\ket{\varphi_i}\} $ base ON di $\mathcal{H}_q$ &  & $\mathcal{A,B} $ compatibili $\Leftrightarrow $ hanno una base di autovettori comuni
\end{tabular}

\noindent Per trovare una base di autovettori comuni (sapendo già che gli operatori commutano, e se entrambi hanno degenerazioni):
\begin{enumerate}
    \item Trovo autovalori e autovettori di $A$ e $B$;
    \item Autovettori associati ad autovalori \textbf{non} degeneri sono automaticamente autovettori comuni;
    \item Per autovettori associati ad autovalori degeneri, faccio la prova (applico $B$ a un autovettore degenere di $A$);
    \item Se è anche autovettore di $B$, sono a posto (è autovettore comune);
    \item Se non lo è:
    \begin{enumerate}
        \item Definisco un nuovo vettore come combinazione lineare degli autovettori della base dell'autospazio degenere in questione;
        \item Impongo che questo nuovo vettore sia autovettore di $B$;
        \item Risolvo il sistema di equazioni trovando i coefficienti della combinazione lineare;
        \item Per come è stato definito, questo vettore è autovettore sia di $A$ che di $B$.
    \end{enumerate}
\end{enumerate}

\noindent Per capire se un insieme di osservabili compatibili costituisce un ICOC:
\begin{enumerate}
    \item Se gli osservabili sono compatibili, esiste una base comune di autovettori;
    \item A ogni autovettore, associo una \textit{label} costituita da una lista dei corrispondenti autovalori per ogni osservabile;
    \item Se \textbf{ogni} \textit{label} è unica, l'insieme è un ICOC.
\end{enumerate}

\section*{Evoluzione temporale e trasformazioni unitarie}

\begin{tabular}{ccccc}
    Equazione di Schrödinger & Trasformazioni unitarie & \multicolumn{2}{c}{Operatore di evoluzione temporale} & Sistema conservativo \\
    $ -i\hbar\frac{\mathrm{d}}{\mathrm{d}t}\ket{\psi(t)} = H(t)\ket{\psi(t)} $ & $\begin{cases}\ket{\psi} &\mapsto \ket{\psi'} = U\ket{\psi} \\ A &\mapsto A' = UAU^\dagger \end{cases}$ & $U(t,t_0)\ket{\psi(t_0)} = \ket{\psi(t)} $ & $U(t+\mathrm{d}t,t) = \id -  \frac{i}{\hbar}H(t)\mathrm{d}t $ & $ U(t,t_0) = e^{-\frac{i}{\hbar}H(t-t_0)} $ \\
    Visuale di Schrödinger & Visuale di Heisenberg &  & Equazione di heisenberg &  \\
    $\begin{cases}\ket{\psi(t)}_S = U(\Delta t) \ket{\psi(t_0)}_S \\ A_S (t) = A_S (0) \end{cases} $ & \multicolumn{2}{l}{$\begin{cases}\ket{\psi(t)}_H = \ket{\psi(t_0)}_H \\ A_H(t) = U^\dagger(\Delta t)A_H(t_0)U(\Delta t) \end{cases} $} & $ i\hbar\frac{\mathrm{d}}{\mathrm{d}t}A_H(t) = [A_H, H] $
\end{tabular}

\section*{Matrice densità}

\begin{tabular}{ccccc}
    Stato puro & Stato misto & \multicolumn{2}{c}{Proprietà (stato generico)} & \textbf{SOLO PER STATI PURI} \\
    $\rho(t) = \ket{\psi(t)}\bra{\psi(t)} $ & $\{\ket{\psi_k(t)}\},\ p_k $ & $\rho^\dagger(t) = \rho(t) $ & $\langle A \rangle_\psi(t) = Tr(\rho(t) A) $ & $\rho^2(t) = \rho(t) $ \\
    $\rho_{pn}(t) = \bra{u_p}\rho(t)\ket{u_n} = \bar{c}_n(t)c_p(t) $ & $\rho(t) = \sum_k p_k\rho_k(t) $ & $Tr(\rho(t)) = 1 $ & $i\hbar\frac{\mathrm{d}\rho(t)}{\mathrm{d}t} = [H(t), \rho(t)] $ & $Tr(\rho^2(t)) = 1 $
\end{tabular}

\section*{Oscillatore armonico}

\begin{tabular}{ccccc}
    Problema agli autovalori & \multicolumn{2}{c}{Definizioni utili} & Nuovo probl. autov. & Autostati \\
    $H\ket{\psi} = E\ket{\psi} $ & $\hat{X} := \sqrt{\frac{m\omega}{\hbar}}X $ & $a = \ngrt{2}(\hat{X}+i\hat{P}) $ & $N\ket{\varphi_\nu^i} = \nu\ket{\varphi_\nu^i} $ & $a\ket{n} = \sqrt{n}\ket{n-1} $ \\
    $H = \frac{P^2}{2m} + \frac{1}{2}m\omega^2X^2 $ & $\hat{P} := \ngrt{m\hbar\omega}P $ & $a^\dagger = \ngrt{2}(\hat{X}-i\hat{P}) $ & $E_\nu = \hbar\omega\left(\nu+\frac{1}{2}\right) $ & $a^\dagger\ket{n} = \sqrt{n+1}\ket{n+1} $ \\
    Autofunzione n-esima & $\hat{H} = \frac{H}{\hbar\omega} = \frac{1}{2}(\hat{X}^2 + \hat{P}^2) $ & $N = a^\dagger a $ & $\nu \in \mathbb{N} $ & $[N,a] = -a,\quad [N,a^\dagger] = +a^\dagger $ \\
    \multicolumn{2}{c}{$\displaystyle u_n(x) = \left[\frac{1}{n!2^n}\left(\frac{\hbar}{m\omega}\right)^n\right]^\frac{1}{2} \left(\frac{m\omega}{\pi\hbar}\right)^\frac{1}{4}\left[\frac{m\omega}{\hbar}x-\frac{\mathrm{d}}{\mathrm{d}x}\right]^n e^{-\frac{m\omega}{\hbar}\frac{x^2}{2}} $} & $[a,a^\dagger] = 1 $
\end{tabular}

\newpage
% qui poi ci mettiamo la teoria

\end{document}
\documentclass{article}
\usepackage{graphicx}
\usepackage{geometry}
\usepackage{tabularx}
\usepackage{float}
\usepackage{amsmath}
\usepackage{amssymb}
\usepackage{textcomp}


\geometry{
    top = 0.5 cm,
    bottom = 0.5 cm,
    left = 0.5 cm,
    right = 0.5 cm
}

% notazione Densa per le radici a denominatore
% (ovviamente va usato in un math environment)
\newcommand{\ngrt}[2][]{
    \sqrt[-#1]{#2}
}

% modo facile per scrivere bra e ket di cose
% (\left e \right servono per l'autosize delle parentesi)
\newcommand{\bra}[1]{
    \left\langle #1 \right|
}
\newcommand{\ket}[1]{
    \left| #1 \right\rangle
}
\newcommand{\bkprod}[2]{
    \left\langle #1 | #2 \right\rangle
}

\begin{document}

Notazione: $\displaystyle \ngrt{x} := \frac{1}{\sqrt{x}} $

\section*{Stati}

\begin{tabular}{cccc}
    Principio 1 & Funzione d'onda e densità di probabilità & Trasformata di Fourier & Basi generalizzate \\
    $\mathcal{S} \mapsto \mathcal{H} $ & $P(x) = \displaystyle \frac{|\psi(x)|^2}{||\psi(x)||^2} $ & $\widetilde{\psi}(p) = \displaystyle \ngrt{2\pi\hbar}\int\mathrm{d}x\psi(x)e^{-\frac{ipx}{\hbar}} $ & $\ket{x} = \xi_x(x) = \delta(x-x_0) $ \\
    $\Sigma \mapsto \hat{\psi} := \{\lambda\ket{\psi} |\, \lambda\in\mathbb{C}\backslash\{0\} \} $ & $P(x) \geq 0,\quad \displaystyle \int \mathrm{d}x P(x) = 1 $ & $P(p) = \displaystyle \frac{|\psi(p)|^2}{||\psi(p)||^2} $ & $\ket{p} = v_p (x) = \ngrt{2\pi\hbar}\,e^{\frac{ipx}{\hbar}} $ \\
     &  &  & $\bkprod{x_0}{x_0'} = \delta(x_0 - x_0') $ \\
     &  &  & $\bkprod{p_0}{p_0'} = \delta(p_0 - p_0') $
\end{tabular}

\section*{Osservabili}

\begin{tabular}{ccccc}
    Posizione e impulso & Principio 2 & Principio 3 &  &  \\
    $X\psi(x) = x\psi(x) $ & $\mathcal{A}\mapsto A $ & $A\ket{a} = a\ket{a} $ &  &  \\
    $P\psi(x) = \displaystyle -i\hbar\frac{\mathrm{d}\psi(x)}{\mathrm{d}x} $ & $\displaystyle\langle \mathcal{A}\rangle_\Sigma = \frac{\bra{\psi}A\ket{\psi}}{\bkprod{\psi}{\psi}} $ & $\sigma(\mathcal{A}) = \sigma(A) $ &  &  \\
    $[X,P] = i\hbar$ & $\Delta A = \sqrt{\langle A^2 \rangle - \langle A \rangle^2} $
\end{tabular}

\newpage



\newpage
% qui poi ci mettiamo la teoria

\end{document}